\documentclass{book}
\usepackage{CJK}

\begin{document}

\begin{CJK}{UTF8}{gbsn}

\author{郑聪}
\title{OpenGL超级宝典}
\maketitle

\chapter{前言}

这本书既适合于通过OpenGL学习计算机图形学的人,也适合于了解计算机图形学但是想要学习OpenGL的人。
本书的目标读者是计算机科学,计算机图形学,或者游戏设计的学生,或者想学计算机图形学的爱好者。
我们并不要求读者了解任何有关计算机图形学或者OpenGL的只是。但是,读者需要了解一些C++的知识。

本书的目标之一就是不要过多的使用参考文献。
这本书的可读性非常强,如果你从头到尾阅读本书,你就会对OpenGL的工作方式和使用方式非常了解。
读完并理解完这本书的内容之后,你就应该有自行阅读最先进的计算机图形学文献,并使用OpenGL实现他们。

涉及OpenGL的所有内容并不是本书的目标之一,本书也不会提到OpenGL的每个函数,或者可以传入到该函数的每个参数。
相反,本书的目标之一就是让你对OpenGL有一个实际的了解。读完本书之后,读者应该可以轻松的了解OpenGL的官方文档中提到的各个函数,
并在他们自己的机器上使用这些函数。

\subsection{本书的结构}
本书大体可以分成三个部分。
在第一部分,我们介绍什么是OpenGL,它如何链接各级图形管线,并给出了一个并不需要太多背景知识的例子。
我们在3D计算机图形学之前简要介绍了一些需要的数学知识。并简要介绍了OpenGL如何处理大规模的数据。
我们也会介绍着色器编程的方法,这将是很多OpenGL程序的核心。

在第二部分,我们开始介绍一些需要多种图形管线知识的OpenGL特性。
在这一部分,我们将会告诉你更加复杂的知识。
这一次重温OpenGL之后,你将会知道,在每一步渲染之后,你的数据去哪了。

在本书的最后一部分,我们将会更加深入的解释图形管线,这一部分将会设计更多高盛的话题,并且会给出更多例子。
我们提供了一些实现了某种渲染技术的例子,并会给出一些使用OpenGL时的建议。
然后,我们会简要介绍各种平台(包括移动平台)上的OpenGL程序。

在第一部分,我们会缓慢的开始,然后快速遍历OpenGL的各种特性。
而后,我们会介绍一些基本知识。在这一张,你将会找到

\(\bullet\)第一章,“简介”,简要介绍OpenGL,它的起源,历史和当前状态.

\(\bullet\)第二章,“第一个OpenGL程序”,将会告诉你如何创建一个简单的OpenGL程序。

\(\bullet\)第三章,“渲染管线”,将会更加细致的了解OpenGL的各个部分。并会扩展之前的例子,用于讲解OpenGL的各个部分。

\(\bullet\)第四章,“3D图形的数学基础”,将会介绍在3D图形学和OpenGL中经常使用的数学知识。

\(\bullet\)第五章,“数据”,将会告诉你如何处理OpenGL需要和它产生的数据。

\(\bullet\)第六章,“着色器和着色器程序”,更加深入的了解\emph{着色器},它们是现代图形学应用的基础。

在Part II,我们将会更加深入的了解第一部分中提到的各种技术。我们将会更加深入的了解OpenGL的各个部分,我们的例子也会更加复杂和有趣。
在这一部分,你会发现:


///////不说了,直接开始

\part{基础}

\chapter{介绍}
\section{你在本章将会学到什么}
\(\bullet\)什么是图形管线,OpenGL怎么使用它。
\(\bullet\)OpenGL的历史。

\end{CJK}

\end{document}
